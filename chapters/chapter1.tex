\chapter{Introduzione}
\label{chapter1}

\section{Cosa sono le smart card}
Una smart card è un dispositivo hardware realizzato su un supporto di plastica in grado di elaborare e memorizzare dati, rispondendo anche ad elevati standard di sicurezza.

La prima idea di realizzare questo tipo di dispositivi è venuta nel 1968 a due inventori tedeschi: Jürgen Dethloff e Helmut Grötrupp. Da allora, grazie alle nuove tecnologie che hanno permesso di realizzare dispositivi sempre più piccoli e potenti, tale tecnologia si è diffusa proprio grazie alla sua versatilità.

Gli utilizzi delle smart card sono molteplici, dal settore bancario, a quello della telefonia, dell'identità, fino ad arrivare nel mondo del trasporto, con l'utilizzo di biglietti elettronici.
\cite{wiki_sc}

\section{Caratteristiche  delle smart card}

Ci sono due criteri che permettono di classificare le smart card. Basandoci sulle potenzialità del circuito possiamo definire le smartcard a sola memoria o a microprocessore. Invece il tipo di interfaccia di collegamento ci permettono di distinguere smartcard a contatto, senza contatto e con antenna e contattiera (ovvero a doppia interfaccia).

Le smart card a sola memoria hanno la capacità di memorizzare informazioni che possono essere lette successivamente, mentre le card a microprocessore possono effettuare elaborazioni. Infine le smarcard a contatto hanno dei pin connettori tramite i quali possono essere alimentate ed è possibile inviare e ricevere informazioni. Mentre quelle senza contatto hanno a disposizione un'antenna che reagisce a un particolare campo elettromagnetico emesso da un dispositivo di lettura/scrittura.

Le smart card rispondono allo standard ISO 7816 che definisce le caratteristiche che devono avere le card a contatto, mentre altri standard sono usati per le card senza contatto (SO 14443 e ISO 15693).

Le smart card a microprocessore sono particolarmente utilizzate per conservare in maniera sicura una chiave privata. Grazie alla loro capacità di elaborazione sono in grado di ricevere una piccola quantità di dati (come ad esempio un hash di un documento) e restituirlo crittografato con la chiave privata contenuta al loro intero. In questo modo la chiave non "esce" mai dal microprocessore, e quindi non può essere letta in alcun modo. Grazie a questa caratteristica le smart card costituiscono un elemento sicuro per la firma digitale di documenti.
\cite{wiki_sc}




    
% \subsection{Sottosezione}
%    \begin{equation}
%    	x_n = c \, x_n(1 - x_n)
%    	\label{Ec:logis}
%    \end{equation}
    
