\chapter*{Introduzione}
\label{introduction}
\addcontentsline{toc}{chapter}{Introduzione}
Quello della sicurezza informatica è un settore molto importante e in continua evoluzione.

La crittografia gioca un ruolo fondamentale in questo campo in quanto la trasmissione dei dati viene spesso effettuata tramite canali poco sicuri e facilmente intercettabili.

I pilastri sui quali si basa la sicurezza, ovvero le problematiche che si cerca di risolvere tramite la crittografia, sono quattro:

\begin{itemize}
    \item \textbf{Riservatezza} di un messaggio, ovvero la garanzia che un messaggio possa essere letto solo dal destinatario, evitando che possa essere intercettato e letto da entità non desiderate.
    \item \textbf{Integrità} del contenuto. Il messaggio inviato deve arrivare al destinatario senza subire modifiche o manipolazione da parte di terzi.
    \item \textbf{Autenticazione} della persona con la quale si sta comunicando. Si vuole garantire che un'entità sia effettivamente ciò che dice di essere.
    \item \textbf{Disponibilità dei servizi}. Questa problematica riguarda in particolare le aziende che hanno server accessibili pubblicamente e che quindi possono essere soggetti di attacchi DoS (\textit{Denial of Service})
\end{itemize}

Per far fronte a queste problematiche (in particolare le prime tre) sono stati sviluppati vari algoritmi di crittografia, come ad esempio algoritmi a chiave simmetrica o a doppia chiave (pubblica e privata).

Come verrà illustrato nel capitolo \ref{chapter1} di questa tesi le smart card rivestono un ruolo fondamentale nel campo della crittografia e sono utilizzate in molti ambiti, come quello della telefonia mobile (paragrafo \ref{applet_sim}), del settore bancario (paragrafo \ref{carta_di_credito}) o nella più recente carta d'identità elettronica (discussa nel paragrafo \ref{carta_identita_elettronica}).

Nel capitolo \ref{chapter2} verrà illustrato più nel dettaglio il funzionamento di una smart card parlando dello standard ISO 7816 al quale risponde nel paragrafo \ref{standard} e accennando alla programmazione delle Java Card nel paragrafo \ref{java_card} e allo Smart Card Web server (paragrafo \ref{smart_card_web_server}).

Il capitolo \ref{chapter3} illustra altre tecnologie utilizzate nel settore della crittografica, in particolare si accennerà al \textit{Trusted Platrofm Module} nel paragrafo \ref{trusted_platform_module}.

