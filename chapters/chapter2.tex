\chapter{Smart card}
\label{chapter2}

%-----------SECTION------------
\section{Lo standard ISO 7816}
\label{standard}

Lo standard ISO 7816 è uno standard internazionale gestito dalla ISO (International Organization for Standardization) e dalla IEC (International Electrotechnical Commission).

Lo standard è composto da varie parti, ognuna delle quali serve per descrivere un dato aspetto delle card, le più importanti sono le prime 4 che descrivono le caratteristiche fisiche ed elettroniche della carta nonché l'organizzazione dei file, la sicurezza e i comandi per lo scambio di informazioni.
\cite{wiki_iso}

\subsection{ISO 7816-1 e ISO 7816-2}
La prima e la seconda parte dello standard descrivono le caratteristiche prettamente fisiche della card, come la dimensione della carta e dei contatti, la resistenza a flessione e piegamento nonché i materiali da utilizzare per la loro fabbricazione. Questo per garantire elevati standard di sicurezza e un tempo di vita adeguato.

Inoltre lo standard definisce anche i limiti di esposizione a raggi X, luce ultravioletta, campi elettromagnetici e temperatura che la carta deve sopportare.

Tutte queste informazioni servono ai costruttori per la fabbricazione delle carte.
\cite{iso}

\subsection{ISO 7816-3}
La terza parte dello standard definisce le caratteristiche elettriche dei contatti e i protocolli di comunicazione con la card. Queste informazioni sono di fondamentale importanza per chi fabbrica lettori o per sviluppatori che vogliono stabilire una comunicazione con il chip presente sulla scheda.
\cite{iso}

In particolare ci sono tre classi di carte a seconda della tensione di alimentazione alla quale lavorano (VCC): Classe A (VCC = 5V), classe B (VCC = 3V) e classe C (VCC = 1.8V).

Sulla card sono presenti 7 contatti, il primo e il quinto sono rispettivamente l'alimentazione (VCC) e il ground (GND), il secondo serve per inviare un segnale di reset, il terzo serve per inviare gli impulsi del clock. Il sesto contatto viene riservato per un utilizzo standard o proprietario come seconda porta I/O. Infine l'ultimo contatto serve per un I/O dei dati in maniera seriale. Il sesto contatto dal 1990 viene spesso utilizzato per fornire alla scheda una tensione di programmazione.
\cite{isoiec3}

\subsection{ISO 7816-4}
La quarta parte dello standard è sicuramente la più interessante per i programmatori che utilizzano un linguaggio più ad alto livello (come ad esempio il Java - paragrafo \ref{java_card}) per realizzare applet, salvare informazioni o far eseguire alcune operazioni dal processore della scheda.

In questa parte dello standard sono indicati i comandi che può ricevere una smart card insieme alla struttura del file system e l'architettura di sicurezza che definisce i diritti di accesso ai dati presenti sulla memoria della card.

Per comunicare con la carta va inviato un comando e attesa una risposta, come indicato dalla terza parte dello standard i bit sono inviati e letti in maniera sequenziale. Un comando è composto da un header e da un corpo. L'header è composto da tre campi: 1 byte di classe denotato CLA, 1 byte di istruzione denotato INS e 2 byte per i parametri denotati P1 e P2 rispettivamente.

All'header seguono una serie di byte per l'invio di eventuali dati alla card. Il primo byte indica il numero di byte che saranno inviati, questo è denominato come L\textsubscript{c}. Questo campo può non essere presente se non vengono inviati dati, oppure occupare fino a 3 byte. Ovviamente subito dopo si hanno gli N byte indicati dal campo L\textsubscript{c} usati per inviare i dati necessari alla carta. Infine il comando si chiude con un campo L\textsubscript{e} che indica il numero massimo di byte che ci si aspetta come risposta dalla carta. Anche questo campo può essere assente o occupare fino a 3 byte.

La risposta è molto più semplice e contiene solo due campi: il primo è un numero di byte al più uguale a quanto indicato dal campo L\textsubscript{e} che contiene gli eventuali dati inviati dalla risposta. Il secondo campo è formato da due byte di stato denominati rispettivamente SW1 e SW2.

\subsubsection{File system}
Come specificato dallo standard ISO 7816-4 i file presenti sulla smart card si dividono principalmente in due categorie: i file dedicati (DF - dedicated files) e i file elementari (EF - elementary files). I primi sono i file delle applicazioni e delle strutture dati, mentre i secondi sono file che contengono dati. I DF possono essere imparentati tra di loro, mentre ciò non è possibile per gli EF. Gli EF sono a loro volta divisi in due categorie gli EF di lavoro e gli EF inteni. I primi sono utilizzati ``dal mondo esterno" mentre i secondi sono dati interni utilizzati dalla carta.

I file sono organizzati in una struttura albero il cui file DF che si trova alla radice viene chiamato master file (MF), come mostrato in figura \ref{fig:file_system}.

\begin{figure}[h!]
  \centering
  \includegraphics[width=400pt]{pictures/gerarchia_file.png}
  \caption{Esempio di una gerarchia di file in una smartcard.}
  \label{fig:file_system}
\end{figure}

\subsubsection{Sicurezza}
L'accesso ai file è, naturalmente, protetto. Lo standard fa riferimento a uno stato di sicurezza ovvero uno stato che è possibile ottenere al seguito di una procedura di autenticazione nella quale viene richiesta, ad esempio, la conoscenza di una password o di una key.

Vengono considerate quattro tipologie di stato di sicurezza:
\begin{itemize}
    \item \textbf{Stato di sicurezza globale}, è solitamente legato a una procedura di autenticazione risiedente nel MF e quindi serve per proteggere l'intera struttura del file system.
    \item \textbf{Stato di sicurezza specifico per un'applicazione}, può essere modificato al seguito del completamento di una procedura di autenticazione basata su un'applicazione, può essere mantenuto, perso o recuperato quando durante la procedura di selezione di un'applicazione. Questo stato è rilevante solo per l'applicazione alla quale fa riferimento.
    \item \textbf{Stato di sicurezza specifico per un file}, il funzionamento è simile allo stato relativo a un'applicazione, tuttavia lo stato si basa su un DF specifico e può essere mantenuto, perso o recuperato quando viene selezionato un file. Uno stato di sicurezza globale può essere visto come un caso particolare.
    \item \textbf{Stato di sicurezza relativo a un comando}, quest'ultimo stato esiste solo quando viene processato un comando usando una comunicazione sicura che richiede un'autenticazione.
\end{itemize}

Una volta stabilito lo stato in cui si trova chi sta comunicando con la carta, è possibile definire gli attributi di sicurezza, questi definiscono le azioni permesse e le condizioni che devono essere verificate per poterle effettuare, inoltre differiscono tra DF e EF e si basano su parametri opzionali presenti in un dato file o nel suo ``padre". In particolare questi attributi specificano in quale stato bisogna trovarsi per poter accedere a un dato file e le funzioni (ad esempio \textit{sola lettura}) che sono disponibili quando ci troviamo in un dato stato.

Ultimo aspetto da considerare sono i meccanismi di sicurezza per l'autenticazione. Prima di tutto troviamo l'autenticazione tramite password o tramite key, per la prima la card compara dati inviati dall'esterno con dati segreti presenti al suo interno, mentre per la seconda viene richiesto a chi vuole accedere alla card di mostrare le conoscenza di una chiave privata.

Un altro meccanismo di sicurezza è quello sull'autenticazione dei dati che consiste nell'utilizzo di una chiave segreta o privata per controllare i dati ricevuti dall'esterno. In alternativa la card può usare sempre una chiave segreta o privata per calcolare una checksum o firma digitale da inviare al lettore. Infine l'ultimo meccanismo disponibile e quello della crittografia dei dati, secondo il quale la scheda decifra o cifra dati scambiati con l'esterno usando una chiave segreta o privata.

Se richiesto dall'applicazione i risultati di un'autenticazione possono essere salvati su un file di log interno (EF).
\cite{isoiec3}

%-----------SECTION------------
\section{Java Card}
\label{java_card}
La tecnologia JavaCard permette di realizzare applicazioni in linguaggio Java, dette applet, da far girare sui chip delle smart card in sicurezza. Questa teclogogia è ampiamente utilizzata nel settore delle SIM (di cui un esempio è riportato nel paragrafo \ref{applet_sim}) e delle carte bancarie.
\cite{Wiki_java}

Con l'ausilio del manuale ufficiale pubblicato da Oracle \cite{javacard3platform} verrà illustrata sinteticamente la piattaforma \textbf{Java Card 3} ovvero un kit di sviluppo per applet Java Card.
\subsection{Architettura della piattaforma}
L'architettura classica della piattaforma è costruita su una classica macchina virtuale java.

Il kit di sviluppo include anche un simulatore (\textit{Java Card RE}) che simula l'ambiente che si avrebbe su una carta fisica e implementa le specifiche dello stadard ISO:7816-4:2013 che è trattato più approfonditamente nel paragrafo \ref{standard}. Il simulatore supporta anche venti canali logici e le estensioni APDU (\textit{Application Protocol Data Units}) definite nello standard ISO 7816-3.
\subsubsection{Java Card TCK}
Nel kit di sviluppo fornito da Oracle viene anche fornita una suite di test automatica e configurabile chiamata \textit{Java Card Technology Compatibility Kit} atta a verificare la compatibilità tra l'applet che si stà sviluppando e le specifiche della carta sulla quale la si vuole far girare.

\subsection{Sviluppo degli applet}
Lo sviluppo delgi applet può essere effettuato tramite l'ide Eclipse installando l'\textit{Eclipse Java Card Plug-in}.

I passaggi per lo sviluppo di un applet sono i seguenti:
\begin{itemize}
    \item Installare e impostare l'ambiente di sviluppo usando l'IDE Eclipse e il Plug-in
    \item Sviluppare l'applet.
    \item Fare il debugging dell'applet.
    \item Creare il file CAP che può essere scaricato sul simulatore o su una card compatibile. Il file viene inviato tramite APDU. Il kit offre un convertitore utilizzabile per generare file CAP da inviare alla scheda (vedere il paragrafo \ref{cap}).
\end{itemize}

\subsection{Utilizzo dei file CAP}
\label{cap}
Per essere installato su una smart card un applet deve essere convertito in Converted Applet (CAP). Per generare questo tipo di file il kit mette a disposizione un convertitore.


Un file CAP utilizza il formato JAR (Java Archive) e, oltre a varie informazioni sul pacchetto Java, contiene un manifesto (\textit{META-INF/MANIFEST.MF}) che fonrnisce una serie di informazioni riguardanti il file. Queste informazioni possono essere usate per facilitare la distribuzione del file.

Le informazioni presenti nel file sono presentate con lo standard \textit{nome:valore} e sono riportate nella tabella \ref{parametri_manifesto}.

\begin{table}[h!]
\centering
\begin{tabular}{ |c|l| } 
 \hline
 \textbf{Nome} &  \textbf{Descrizione del valore} \\
 \hline
 Java-Card-Creation-Time &  Indica quando è stato generato il file CAP \\
 \hline
 Java-Card-Converter-Version & Indica la versione del convertitore utilizzata \\ 
 \hline
 Java-Card-Converter-Provider & Indica il fornitore del convertitore \\ 
 \hline
 Java-Card-File-Version & Versione del file CAP \\
  & secondo la convenzione \textit{maggiore.minore} \\ 
 \hline
 Java-Card-Package-Version & Versione del pacchetto file CAP\\
  & secondo la convenzione \textit{maggiore.minore} \\ 
 \hline
 Java-Card-Package-AID & AID (\textit{JADE Agent Identifier}) del pacchetto \\
 \hline
 Java-Card-Package-Name & Il nome del pacchetto \\
 \hline
 Java-Card-Applet-$<$n$>$-AID & AID (\textit{JADE Agent Identifier}) dell'applet \textit{n}\\
 \hline
 Java-Card-Applet-$<$n$>$-Name & Il nome breve della classe \textit{n} \\
  & implementata dal CAP \\
 \hline
 Java-Card-Import-Package-$<$n$>$-AID & AID (\textit{JADE Agent Identifier})\\
  & del pacchetto \textit{n} importato \\
 \hline
 Java-Card-Import-Package-$<$n$>$-Version & Versione del pacchetto \textit{n} importato\\
  & secondo la convenzione \textit{maggiore.minore}\\
 \hline
 Java-Card-Integer-Support-Required & Può assumere valori \textit{TRUE} o \textit{FALSE}.\\
  & Il valore vero indica che il pacchetto\\
   & richiede l'integer support \\
 \hline
 
\end{tabular}
\caption{Tabella dei parametri presenti nel manifesto del file.}
\label{parametri_manifesto}
\end{table}

\subsubsection{Generazione di un file CAP}
Per generare un file CAP è possibile utilizzare il tool \textit{capgen} fornito dal kit.

Per avviare il tool è possibile usare il comando
\begin{center}
    \textit{capgen.bat \textbf{[opzioni] nome\_del\_file}}.
\end{center}

Le opzioni disponibili sono:
\begin{itemize}
    \item \textbf{-help} stampa un messaggio di aiuto.
    \item \textbf{-nobanner} sopprime i vari messaggi.
    \item \textbf{-o \textit{nome\_del\_file}} permette di specificare un file di output.
\end{itemize}
 
\subsection{Fare il debugging delle applicazioni}
Per effettuare il debugging delle applicazioni il kit offre due tool che lavorano insieme per alleggerire la procedura, altrimenti troppo impegnativa per la virtual machine del simulatore.

Il primo tool, \textit{cref} può essere lanciato direttamente da Eclipse o da riga di comando e ha la possibilità di simulare la memoria persistente (EEPROM) nonché di salvare e recuperare i dati salvati sulla memoria o su file presenti sull'hard disk. Inoltre il tool può eseguire operazioni di I/O tramite un'interfaccia socket che simula il collegamento tra il lettore di carte e il computer.

Per il debugging l'IDE utilizza il Java Debug Wire Protocol (JDWP) che, come accennato, è troppo pesante per la piccola VM utilizzata dal simulatore fornito dal kit. Per questo per il debugging viene utilizzato un protocollo proprietario più leggero.

Il secondo tool offerto è il \textit{debugproxy}, esso ha il compito di tradurre comandi e risposte tra l'IDE e il simulatore \textit{cref} utilizzando il protocollo appropriato.

Dato che i tool \textit{cref}, \textit{debugproxy} e l'IDE comunicano tramite socket, essi possono girare su host diversi.

Da Eclipse è possibile far partire il debug proxy per impostare breakpoints, leggere o impostare variabili e fare il debug di una libreria.

\subsection{Distribuire le proprie applicazioni}
Il kit permette di scaricare un Java Card technology package, effettuare il collegamento con la card, eliminare applets e pacchetti dalla smart card e impostare le applet di default per i vari canali logici.

Per installare l'applicazione su una card bisogna prima convertire i file .class in file .cap utilizzando il convertitore fornito dal kit, successivamente il tool \textit{scriptgen}, ovvero l'off-card installer converte il file .cap in uno script .scr che consiste in una serie di comandi APDU che vengono eseguiti dall'APDUtool. Infine l'on-card installer processa il file CAP e invia gli APDU di risposta all'APDUtool con lo stato ed eventuali dati.

Il file .scr contiene comandi e C-APDU che sono terminati da un punto e virgola.
La sintassi di un C-APDU è la seguente:
\begin{center}
    $<$CLA$>$ $<$INS$>$ $<$P1$>$ $<$P2$>$ $<$LC$>$ [$<$byte 0$>$...$<$byte LC-1$>$] $<$LE$>$;
\end{center}
dove
\begin{itemize}
    \item $<$CLA$>$ è il byte della classe definito dallo standard ISO 7816-4.
    \item $<$INS$>$ è il byte di istruzione definito dallo standard ISO 7816-4.
    \item $<$P1$>$, $<$P2$>$ sono i parametri P1 e P2 definiti dallo standard ISO 7816-4.
    \item $<$LC$>$ è il numero dei byte inviati (1 in modalità non estesa, 2 in modalità estesa).
    \item $<$byte 0$>$...$<$byte LC-1$>$ sono i byte per i dati di input.
    \item $<$LE$>$ è la lunghezza che ci si aspetta per l'output (1 in modalità non estesa, 2 in modalità estesa).
\end{itemize}

Il protocollo utilizzato per installare un applet è composto da una sequenza di comandi ben precisa. Per prima cosa viene inviato un comando di selezione utilizzato per invocare l'on-card installer, segue un comando di CAP Begin. Successivamente viene ripetuta una serie di 3 comandi per ogni componente presente nel file CAP (Component \#\# Begin, Component \#\# Data, Component \#\# End). La sequenza viene conclusa dai comandi CAP End e Create Applet. Ogni comando viene inviato alla card e riceve una risposta che varia a seconda del comando.

%----------------------------------------
\section{Funzionalità avanzate}
La tecnologia della smart card si è evoluta nel tempo, grazie anche ai nuovi sviluppi tecnologici che hanno portato alla realizzazione di circuiti integrati sempre più piccoli e veloci. Ciò ha permesso lo sviluppo di smart card sempre più potenti capaci di far girare programmi sempre più complessi.

\subsection{Smart Card Web Server}
\label{smart_card_web_server}
Uno \textit{Smart Card Web Server} (SCWS) è un server HTTP implementato all'interno di una smart card, di solito integrata in uno smartphone (SIM - vedi paragrafo \ref{applet_sim}). Questa tecnologia permette agli operatori di telefonia mobile di offrire determinati servizi utilizzando il diffusissimo protocollo \textit{HTTP/1.1}.

Il principale obiettivo di questa tecnologia è quello di creare una comunicazione interna al dispositivo tra un WEB browser che gira nello smartphone e un server presente sulla card. Permette, inolte, un'amministrazione remota della smart card da un'entità autorizzata (ad esempio il fornitore del servizio di telefonia e/o della card stessa).

L'interfaccia HTTP offerta dalla card si trova a un livello logico separato dall'interfaccia di comunicazione definita dallo standard ISO 7816 (vedi paragrafo \ref{standard}) e permette ad applicazioni HTTP di comunicare con la card in maniera autonoma.

Questo canale di comunicazione è utilizzato dai fornitori per offrire maggiori servizi ai loro clienti.

L'URL utilizzato per la comunicazione deve corrispondere alla definizione data dal protocollo \textit{HTTP/1.1}, ovvero

\begin{center}
    http\_URL = ``http:" ``//" host [ ``:" port ] [ abs\_path [ ``?" query ]]
\end{center}

La $<query>$ opzionale deve essere una sequenza di uno o più termini della forma $<name>=<value>$ separati da un '\&'. Il server deve poter rispondere perlomeno a url di 1024 caratteri e \textit{abs\_path} di 128 caratteri.

% \subsubsection{I protocolli di trasporto}
% Se la card non ha il proprio indirizzo IP e quindi non supporta il protocollo TCP/IP viene utilizzato il protocollo \textit{ Bearer Independent protocol} (BIT). In particolare, nel client viene fatto girare un gateway BIT che effettua una "traduzione" tra i due protocolli (TCP/IP e BIT). Il protocollo BIT è utilizzato per comunicare tra il gateway e la scheda.

% L'IP utilizzato per accedere al gateway è 127.0.0.1, meglio come conosciuto come localhost. Il gateway BIP ascolta sulle porte definite dallo SCWS, solitamente si usano le porte 3516 e 4116, la prima è riservata come "porta della smart card", la seconda come "smart card TLS".

% Un esempio di url che può essere utilizzato per accedere a una risorsa chiamata "foobar.xhtml" e presente nel percorso "pub/files" è il seguente:
% \begin{center}
%     http://127.0.0.1:3516/pub/files/foobar.xhtml 
% \end{center}

% Mentre il seguente url può essere utilizzato per avviare un'applicazione e passarle dei parametri:
% \begin{center}
%     http://127.0.0.1:4116/cgi/SSO?account=username\&otherparam=123 
% \end{center}

% Se la smart card ha il proprio indirizzo IP e quindi supporta il protocollo TCP/IP allora il dispositivo mobile ha la possibilità di comunicare direttamente con la card senza dover passare per un gatway BIT.

% La connessione è definita dallo standard ETSI TS 102 483]. L'indirizzo IP per comunicare con la card può essere dinamico, tuttavia ci si riferisce chiamandolo “localuicc”. Le porte per la comunicazione sono la 80 per l'HTTP e la 443 per L'HTTP con TLS (presentato le paragrafo \ref{tls}).

% Un paio di URL di esempio sono riportati in seguito:
% \begin{center}
%     http://localuicc/pub/files/foobar.xhtml
%     https://localuicc/cgi/display?df=7F01\&ef=3F01\&record=01\&offset=50\&length=10 
% \end{center}

\subsubsection{Sicurezza della comunicazione}
\label{tls}
Per la sicurezza della trasmissione dei dati con la card viene utilizzato il protocollo \textit{Transport Layer Security}  (TLS). Questo protocollo fornisce un meccanismo sicuro ed affidabile per il trasporto dei dati tra due entità, con un controllo dell'integrità e della confidenzialità delle informazioni scambiate. Fornisce anche dei meccanismi di autenticazione per una o entrambe le parti.

Il TLS è pesato per un paradigma Client-Server dove il client è chi inizia la comunicazione (o invia una richiesta) e il server fornisce una risposta. Solitamente il client può autenticare il server utilizzando un certificato a chiave pubblica. Un'auntenticazione mutua può essere effettuata utilizzando dei certificati a chiave pubblica pre-condivisi (\textit{Pre Shared Keys-TLS} o PSK-TLS)
\cite{scwebserver}

%------------------------------
\subsection{Applicazione SIM}
La SIM card (introdotta nel paragrafo \ref{applet_sim}) oltre a conservare il codice identificativo dell'utente offre allo stesso anche alcune funzionalità, come ad esempio il salvataggio di contatti telefonici, utile quando si cambia cellulare e non si vuole perdere la lista di contatti salvata in rubrica.

Inoltre alcuni operatori offrono delle semplici applicazioni presenti all'interno delle loro SIM card, che presentano un'interfaccia molto semplice (a menù) con alcune funzioni che possono essere richieste dall'utente. Queste applicazioni possono anche aprire URL, mandare SMS, avviare chiamate e reagire a particolari eventi come l'arrivo di una chiamata o la disconnesione di una chiamate (per inviare, ad esempio, un SMS all'utente con il credito residuo o i minuti ancora disponibili nell'offerta). Inoltre c'è la possibilità di interagire con altre card, nel caso di telefoni dual SIM.

\begin{figure}[h!]
  \centering
  \includegraphics[width=100pt]{pictures/icona_aplet_fastweb.jpg}
  \caption{Icona dell'applet della SIM di Fastweb.}
  \label{fig:app_icon}
\end{figure}

Nella figura \ref{fig:app_icon} è possibile vedere l'icona dell'applet fornito dall'operatore Fastweb, mentre nella figura \ref{fig:app_screen} è possibile vederne la schermata principale.

\begin{figure}[h!]
  \centering
  \includegraphics[width=150pt]{pictures/screen_aplet_fastweb.jpg}
  \caption{Schermata dell'applet della SIM di Fastweb.}
  \label{fig:app_screen}
\end{figure}

Il modo più semplice per realizzare applicazioni che girano su smart card (più propriamente dette applet) è quello di utilizzare il liguaggio Java e le Java Cad, di cui si parlerà più nel dettaglio nel paragrafo \ref{java_card}.
\cite{secret_life}

\subsection{Applet firewall}
All'interno di una smart card con sistema operativo Java card è possibile installare più applet diversi che convivono sulla card. In base alla funzionalità implementata da ogni applet è possibile che debba salvare e manipolare dati sensibili, come una valuta virtuale, delle impronte digitali o chiavi di crittografia.

Per questo motivo bisogna fare in modo che un applet non possa accedere ai dati delle altre applicazioni che convivono sulla scheda. Un applet firewall ha il compito di confinare ogni applet in una porzione della memoria e bloccare eventuali accessi ad aree designate alle altre applicazioni.

Il firewall permette, comunque, un meccanismo di condivisione di alcune risorse in maniera controllata e sicura; questo meccanismo influisce sulla programmazione di un applet e va tenuto conto dagli sviluppatori.

Il meccanismo di firewall permette di evitare che errori della programmazione o del design di un applet possano portare alla "fuoriuscita" di dati sensibili. Inoltre il firewall perviene anche tentativi di hacking, in quanto se un'applicazione malevola viene installata su una smart card e riesce ad ottenere il riferimento pubblico di un oggetto che appartiene ad un'altra applicazione, nel momento in cui tenta di accederci il firewall la bloccherà. In questo modo non viene impedita la regolare esecuzione dell'applet ``lecito".

Il firewall divide i vari oggetti in \textit{contesti}, ovvero aree di memoria protette. Il bordo di ogni contesto è costituito dal firewall stesso. Quando un'applicazione viene eseguita, riceve un contesto (detto \textit{group context}). Se ci sono più istanze di applicazioni che fanno parte dello stesso package, allora hanno in assegnazione lo stesso contesto e possono condividere gli oggetti che vi appartengono.

Inoltre il JCRE (\textit{Java Card Runtime Environment}) ha il proprio contesto, ovvero un contesto di sistema con privilegi speciali. Il JCRE può accedere, ad esempio, a tutti gli altri contesti, ma il viceversa è proibito.

La figura \ref{fig:context} mostra un esempio di una possibile divisione dei contesti tra i vari applet.
\cite{applet_firewall}

\begin{figure}[h!]
  \centering
  \includegraphics[width=400pt]{pictures/context.png}
  \caption{Schema logico delle divisione dei contesti in una Java Card.}
  \label{fig:context}
\end{figure}
