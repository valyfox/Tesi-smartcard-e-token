\chapter{Smartcard}
\label{chapter2}

%-----------SECTION------------
\section{Lo standard ISO 7816}
\label{standard}

%-----------SECTION------------
\section{Java Card}
\label{java_card}
La tecnologia JavaCard permette di realizzare applicazioni in linguaggio Java, dette applet, da far girare sui chip delle smart card in sicurezza. Questa teclogogia è ampiamente utilizzata nel settore delle SIM (di cui un esempio è riportato nel paragrafo \ref{applet_sim}) e delle carte bancarie.
\cite{Wiki_java}

Con l'ausilio del manuale ufficiale pubblicato da Oracle \cite{javacard3platform} verrà illustrata sinteticamente la piattaforma \textbf{Java Card 3} ovvero un kit di sviluppo per applet Java Card.
\subsection{Architettura della piattaforma}
L'architettura classica della piattaforma è costruita su una classica macchina virtuale java.

Il kit di sviluppo include anche un simulatore (\textit{Java Card RE}) che simula l'ambiente che si avrebbe su una carta fisica e implementa le specifiche dello stadard ISO:7816-4:2013 che è trattato più approfonditamente nel paragrafo \ref{standard}. Il simulatore supporta anche venti canali logici e le estensioni APDU (\textit{Application Protocol Data Units}) definite nello standard ISO 7816-3.
\subsubsection{Java Card TCK}
Nel kit di sviluppo fornito da Oracle viene anche fornita una suite di test automatica e configurabile chiamata \textit{Java Card Technology Compatibility Kit} atta a verificare la compatibilità tra l'applet che si stà sviluppando e le specifiche della carta sulla quale la si vuole far girare.

\subsection{Sviluppo degli applet}
Lo sviluppo delgi applet può essere effettuato tramite l'ide Eclipse installando l'\textit{Eclipse Java Card Plug-in}.

I passaggi per lo sviluppo di un applet sono i seguenti:
\begin{itemize}
    \item Installare e impostare l'ambiente di sviluppo usando l'IDE Eclipse e il Plug-in
    \item Prendere familiarità con gli esempi.
    \item Sviluppare l'applet.
    \item Fare il debugging dell'applet.
    \item Creare il file CAP che può essere scaricato sul simulatore o su una card compatibile. Il file viene inviato tramite APDU. Il kit offre un convertitore utilizzabile per generare file CAP da inviare alla scheda (vedere il paragrafo \ref{cap}).
\end{itemize}

\subsection{Utilizzo dei file CAP}
\label{cap}
Per essere installato su una smart card un applet deve essere convertito in Converted Applet (CAP). Per generare questo tipo di file il kit mette a disposizione un convertitore.


Un file CAP utilizza il formato JAR (Java Archive) e, oltre a varie informazioni sul pacchetto Java, contiene un manifesto (\textit{META-INF/MANIFEST.MF}) che fonrnisce una serie di informazioni riguardanti il file. Queste informazioni possono essere usate per facilitare la distribuzione del file.

Le informazioni presenti nel file sono presentate con lo standard \textit{nome:valore} e sono riportate nella tabella \ref{parametri_manifesto}.

\begin{table}[h!]
\centering
\begin{tabular}{ |c|l| } 
 \hline
 \textbf{Nome} &  \textbf{Descrizione del valore} \\
 \hline
 Java-Card-Creation-Time &  Indica quando è stato generato il file CAP \\
 \hline
 Java-Card-Converter-Version & Indica la versione del convertitore utilizzata \\ 
 \hline
 Java-Card-Converter-Provider & Indica il fornitore del convertitore \\ 
 \hline
 Java-Card-File-Version & Versione del file CAP \\
  & secondo la convenzione \textit{maggiore.minore} \\ 
 \hline
 Java-Card-Package-Version & Versione del pacchetto file CAP\\
  & secondo la convenzione \textit{maggiore.minore} \\ 
 \hline
 Java-Card-Package-AID & AID (\textit{JADE Agent Identifier}) del pacchetto \\
 \hline
 Java-Card-Package-Name & Il nome del pacchetto \\
 \hline
 Java-Card-Applet-$<$n$>$-AID & AID (\textit{JADE Agent Identifier}) dell'applet \textit{n}\\
 \hline
 Java-Card-Applet-$<$n$>$-Name & Il nome breve della classe \textit{n} \\
  & implementata dal CAP \\
 \hline
 Java-Card-Import-Package-$<$n$>$-AID & AID (\textit{JADE Agent Identifier})\\
  & del pacchetto \textit{n} importato \\
 \hline
 Java-Card-Import-Package-$<$n$>$-Version & Versione del pacchetto \textit{n} importato\\
  & secondo la convenzione \textit{maggiore.minore}\\
 \hline
 Java-Card-Integer-Support-Required & Può assumere valori \textit{TRUE} o \textit{FALSE}.\\
  & Il valore vero indica che il pacchetto\\
   & richiede l'integer support \\
 \hline
 
\end{tabular}
\caption{Tabella dei parametri presenti nel manifesto del file.}
\label{parametri_manifesto}
\end{table}

\subsubsection{Generazione di un file CAP}
Per generare un file CAP è possibile utilizzare il tool \textit{capgen} fornito dal kit.

Per avviare il tool è possibile usare il comando
\begin{center}
    \textit{capgen.bat \textbf{[opzioni] nome\_del\_file}}.
\end{center}

Le opzioni disponibili sono:
\begin{itemize}
    \item \textbf{-help} che stampa un messaggio di aiuto
    \item \textbf{-nobanner} che sopprime i vari messaggi
    \item \textbf{-o \textit{nome\_del\_file}}.
\end{itemize}
 
\subsection{Fare il debugging delle applicazioni}
Per effettuare il debugging delle applicazioni il kit offre due tool che lavorano insieme per alleggerire la procedura, altrimenti troppo impegnativa per la virtual machine del simulatore.

Il primo tool, \textit{cref} può essere lanciato direttamente da Eclipse o da riga di comando e ha la possibilità di simulare la memoria persistente (EEPROM) nonché di salvare e recuperare i dati salvati sulla memoria o su file presenti sull'hard disk. Inoltre il tool può eseguire operazioni di I/O tramite un'interfaccia socket che simula il collegamento tra il lettore di carte e il computer.

Per il debugging l'IDE utilizza il Java Debug Wire Protocol (JDWP) che, come accennato, è troppo pesante per la piccola VM utilizzata dal simulatore fornito dal kit. Per questo per il debugging viene utilizzato un protocollo proprietario più leggero.

Il secondo tool offerto è il \textit{debugproxy}, esso ha il compito di tradurre comandi e risposte tra l'IDE e il simulatore \textit{cref} utilizzando il protocollo appropriato.

Dato che i tool \textit{cref}, \textit{debugproxy} e l'IDE comunicano tramite socket, essi possono girare su host diversi.

Da Eclipse è possibile far partire il debug proxy per impostare breakpoints, leggere o ipostare variabili e fare il debug di una libreria.

\subsection{Distribuire le proprie applicazioni}
Il kit permette di scaricare un Java Card technology package, effettuare il collegamento con la card, eliminare applets e pacchetti dalla smart card e impostare le applet di default per i vari canali logici.

Per installare l'applicazione su una card bisogna prima convertire i file .class in file .cap utilizzando il convertitore fornito dal kit, successivamente il tool \textit{scriptgen}, ovvero l'off-card installer converte il file .cap in uno script .scr che consiste in una serie di comandi APDU che vengono eseguiti dall'APDUtool. Infine l'on-card installer processa il file CAP e invia gli APDU di risposta all'APDUtool con lo stato ed eventuali dati.

Il file .scr contiene comandi e C-APDU che sono terminati da un punto e virgola.
La sintassi di un C-APDU è la seguente:
\begin{center}
    $<$CLA$>$ $<$INS$>$ $<$P1$>$ $<$P2$>$ $<$LC$>$ [$<$byte 0$>$...$<$byte LC-1$>$] $<$LE$>$;
\end{center}
dove
\begin{itemize}
    \item $<$CLA$>$ è il byte della classe definito dallo standard ISO 7816-4.
    \item $<$INS$>$ è il byte di istruzione definito dallo standard ISO 7816-4.
    \item $<$P1$>$, $<$P2$>$ sono i parametri P1 e P2 definiti dallo standard ISO 7816-4.
    \item $<$LC$>$ è il numero dei byte inviati (1 in modalità non estesa, 2 in modalità estesa).
    \item $<$byte 0$>$...$<$byte LC-1$>$ sono i byte per i dati di input.
    \item $<$LE$>$ è la lunghezza che ci si aspetta per l'output (1 in modalità non estesa, 2 in modalità estesa).
\end{itemize}

Il protocollo utilizzato per installare un applet è composto da una sequenza di comandi ben precisa. Per prima cosa viene inviato un comando di selezione utilizzato per invocare l'on-card installer, segue un comando di CAP Begin. Successivamente viene ripetuta una seria di 3 comandi per ogni componente presente nel file CAP (Component \#\# Begin, Component \#\# Data, Component \#\# End). La sequenza viene conclusa dai comandi CAP End e Create Applet. Ogni comando viene inviato alla card e riceve una risposta che varia a seconda del comando.