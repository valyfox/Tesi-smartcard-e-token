\chapter{Smartcard}
\label{chapter2}

%-----------SECTION------------
\section{Java Card}
\label{java_card}
La tecnologia JavaCard permette di realizzare applicazioni in linguaggio Java, dette applet, da far girare sui chip delle smart card in sicurezza. Questa teclogogia è ampiamente utilizzata nel settore delle SIM (di cui un esempio è riportato nel paragrafo \ref{applet_sim}) e delle carte bancarie.
\cite{Wiki_java}

Con l'ausilio del manuale ufficiale pubblicato da Oracle \cite{javacard3platform} verrà illustrata sinteticamente la piattaforma \textbf{Java Card 3} ovvero un kit di sviluppo per applet Java Card.
\subsection{Architettura della piattaforma}
L'architettura classica della piattaforma è costruita su una classica macchina virtuale java.

Il kit di sviluppo include anche un simulatore (\textit{Java Card RE}) che simula l'ambiente che si avrebbe su una carta fisica e implementa le specifiche dello stadard ISO:7816-4:2013 che è trattato più approfonditamente nel paragrafo \ref{standard}. Il simulatore supporta anche venti canali logici e le estensioni APDU (\textit{Application Protocol Data Units}) definite nello standard ISO 7816-3.
\subsubsection{Java Card TCK}
Nel kit di sviluppo fornito da Oracle viene anche fornita una suite di test automatica e configurabile chiamata \textit{Java Card Technology Compatibility Kit} atta a verificare la compatibilità tra l'applet che si stà sviluppando e le specifiche della carta sulla quale la si vuole far girare.

\subsection{Sviluppo degli applet}
Lo sviluppo delgi applet può essere effettuato tramite l'ide Eclipse installando l'\textit{Eclipse Java Card Plug-in}.

I passaggi per lo sviluppo di un applet sono i seguenti:
\begin{itemize}
    \item Installare e impostare l'ambiente di sviluppo usando l'IDE Eclipse e il Plug-in
    \item Prendere familiarità con gli esempi.
    \item Sviluppare l'applet.
    \item Fare il debugging dell'applet.
    \item Creare il file CAP che può essere scaricato sul simulatore o su una card compatibile. Il file viene inviato tramite APDU. Il kit offre un convertitore utilizzabile per generare file CAP da inviare alla scheda.
\end{itemize}


%-----------SECTION------------
\section{Lo standard ISO 7816}

\label{standard}
